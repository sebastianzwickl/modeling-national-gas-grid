\section{State-of-the-art and progress beyond}\label{stateoftheart}
This section discusses the relevant scientific literature within the scope of this work. Three main strands of the literature are covered. First, Section \ref{import} deals with the global and cross-country dimension of natural and renewable gas trade. It focuses on the impact of the energy system decarbonization on gas markets and discusses also intra-country gas supply with a high spatial granularity of a grid representation. Then, Section \ref{approaches} examines different fundamental approaches of modeling gas grids. Section \ref{tariffs} elaborates on the regulation of gas grids and especially on gas grid charges. Building on this discussion of the existing literature, Section \ref{novelties} highlights the novelties and the progress beyond the state of the art of this work.

\subsection{Decarbonized gas markets and cross-country trade}\label{import}
The focus of this section is on how the shift toward decarbonizing energy systems is affecting renewable gas markets. Before delving into the relevant literature, it may be helpful to highlight some key studies on fossil natural gas markets, as these studies provide a comprehensive background for the emerging renewable gas markets, both in terms of current dynamics and historical context. The fundamentals of natural gas markets are described comprehensively by Hulshof et al. \cite{hulshof2016market}. A comprehensive introduction to the historical developments and global trends in natural gas is given by  Balat \cite{balat2009global}. Egging and Gabriel \cite{egging2006examining} analyze the global natural gas trade while focusing on the European natural gas market. Geng et al. \cite{geng2014dynamic} elaborate on the global natural gas market dynamics. Similarly, Esmaeili et al. \cite{esmaeili2022system} study the dynamics of the natural gas market, but with a special focus on renewable energy resources. Going even further into renewable energy resources, Horsching et al. \cite{horschig2018decarbonize} present a dynamic natural gas market model for integrating renewable gases.\vspace{0.3cm}

In 2021, the European Commission has published a proposal for a framework of renewable and natural gases and hydrogen \cite{regulation_renewable_gases}. The aim is to support renewable and low-carbon gases (i.e., biogas, biomethane, renewable and low-carbon hydrogen, and synthetic methane) in Europe and to reach a share of two-thirds of gaseous fuels in the 2050 energy mix. The discussion of renewable gas markets is further elaborated below, where further details on the definition of renewable and low carbon gases can be found in \cite{briefing_renewable_gases}. \added{Shirizadeh and Quirion \cite{shirizadeh2022importance} study the role of renewable gas in France's future low-carbon energy system.} Kolb et al. \cite{kolb2021scenarios} focus in their work on the integration of renewable gases into gas markets, while the remaining one-third of gaseous fuels in 2050 is expected to be still fossil natural gas, but in combination with carbon capture, storage, and utilization. With Bertasini et al. \cite{bertasini2023decarbonization} giving a critical overview of the contribution of renewable gases to the decarbonization of the European energy system and grids, today, renewable and low carbon gases have only a minor contribution to Europe's energy mix. The study by Kolb et al. \cite{kolb2021scenarios} above further provides a comprehensive literature review on renewable gases. Lochner \cite{lochner2011identification} elaborates on the European gas market and the identification of congestions in the gas transmission grid. Gorre et al. \cite{gorre2019production} deal exhaustively with future renewable gas generation costs.\vspace{0.3cm}

A key role in the transition to renewable and low carbon gas markets has the existing gas infrastructure. On the one hand, and in the context of the recently extended terminal capacities for liquified natural gas (LNG) also being worth a mention, the repurposing of existing pipelines, especially at the transmission grid level, allows the buildup of a hydrogen grid, as proposed in the so-called "Hydrogen Backbone" \cite{hydrogen_backbone}. While in the mid-term, the LNG terminals can be used to import renewable and low carbon gases, supporting the European gas market \cite{al2022emerging}. In the short-term, these terminals are used to support Russian natural gas import substitution by fossil LNG imports from exporter countries, such as the United States and Quatar \cite{brauers2021liquefied}. On the other hand, the area-wide existing pipelines of the distribution grid levels (high-, mid-, and low-pressure pipelines) allow the injection of distributed renewable and low carbon gas generation \cite{cucchiella2018profitability}. The following references list which key areas are covered. Sulewski \cite{sulewski2023development} explore the biomethane market in Europe. Schlund and Schönfisch \cite{schlund2021analysing} analyze the impact of renewable quota on the European natural gas markets. Paturska et al. \cite{paturska2015economic} provide an economic assessment of biomethane supply system based on the natural gas grid. Khatiwada \cite{khatiwada2022decarbonization} elaborate on barriers of the decarbonization of natural gas systems. Stürmer \cite{sturmer2020greening} examines in detail on the potentials of renewable gas injection into existing gas grids. Padi et al. \cite{padi2023techno} study the techno-economic potentials of integrating decentralized biomethane generation into existing natural gas grids. \added{A similar study on technologies for injecting biomethane into existing natural gas grids is provided by Kabeyi et al. \cite{kabeyi2022technologies}.}

\subsection{Gas grid modeling approach (top-down and bottom-up)}\label{approaches}
The following literature review focuses on the modeling of natural gas transport by grids and pipelines. The interested reader is referred to Thomas and Dawe \cite{thomas2003review} for a comprehensive review of the options for transporting natural gas, which includes other methods. In general, in dimensions, along with others such as the sectoral dimension (whether or not hydrogen is accounted for in detail), determining the level of consideration given to various factors such as flow conditions of natural gas, pressure levels and drops in transport pipelines, and the operational energy and costs associating with compressors, the literature on gas grid modeling approaches can be divided based on two key dimensions. These are: (i) modeling perspective (e.g., techno-economic), and (ii) spatial scale.\vspace{0.3cm}

A review on optimization of natural gas transportation systems is given by R{\'\i}os-Mercado and Borraz-S{\'a}nchez \cite{rios2015optimization}. It encompasses both transmission and distribution grids. Pfetsch et al. \cite{pfetsch2015validation} elaborate in detail on the operation of gas transmission grids. Pambour et al. \cite{pambour2016integrated} propose an integrated transient model approach for simulating the operation of transmission grids. The transient process in transmission grids is further examined by Liu \cite{liu2011coordinated}. Riepin et al. \cite{riepin2022adaptive} develop in their study an adaptive robust optimzation model for transmission grid expansion planning. Chiang and Zavala \cite{chiang2016large} investigate the interconnection between gas and power transmission grids. O'Donoghue et al. \cite{o1997development} examine transmission pipelines' resistance to high-pressure levels. Liu et al. \cite{liu2009security} study aspects of supply security in detail. A study with a focus on supply security for Europe's natural gas demand is given by Sutrisno and Alkemade \cite{sutrisno2020eu}.\vspace{0.3cm}

With regard to the distribution grid level, Herr{\'a}n-Gonz{\'a}lez et al. \cite{herran2009modeling} provide a comprehensive review on the modeling and simulation of gas grids. Barati et al. \cite{barati2014multi} propose an integrated framework for grid expansion planning.  Giehl et al. \cite{giehl2023assessment} examine the impact of the decarbonization on gas distribution grids. Zwickl-Bernhard and Auer \cite{zwickl2022demystifying} present alternative supply options to natural gas distribution grids. Keogh et al. \cite{keogh2022gas} review technical and modeling studies of renewable gas generation and injection into the distribution grid. The same authors present also a techno-economic case study for renewable gas injection into the distribution grid in \cite{keogh2022gas}. \added{Pellegrino et al. \cite{pellegrino2017greening} study the injection of renewable gas into the transmission grid.} Abeysekera et al. \cite{abeysekera2016steady} analyze the injection of renewable gas in low-pressure gas grids from a technical perspective in detail. Mertins et al. \cite{mertins2023competition} examine the competition between renewable gas and hydrogen injection into distribution grids. Repurposing of natural gas pipelines for hydrogen transport is assessed by Cerniauskas et al. \cite{cerniauskas2020options} and Jayanti \cite{jayanti2022repurposing}. An overview of the modeling of hydrogen grids is given by Reuß et al. \cite{reuss2019modeling}.\vspace{0.3cm}

Finally, the modeling contributions of the open-source community subject of gas grids are discussed. In principle, in a trend that is also continuing in the area of gas grids, open-source approaches are becoming increasingly important in energy system analysis \cite{hulk2018transparency}. For instance, Schmidt et al. \cite{schmidt2017gaslib} provide a set of publicly available gas grid instances that can be used by researchers in the field of gas transport. Pluta et al. \cite{pluta2022scigrid_gas} present an approach for developing an open-source model of the gas transport grid in Europe. Nevertheless, data, with isolated exceptions, e.g., for the transmission grid (see \cite{entsog} for open-source data on the European transmission gas grid) or for the Belgian gas grid in \cite{de2000gas}, on natural gas grids in particular are rarely made publicly available. However, there is often an advantage for those who have this information (e.g., gas grid operators) to scientific researchers and other third parties, particularly with analyses at the distribution grid level. 

\subsection{Regulatory of decarbonized gas grids}\label{tariffs}
Not much has been published on how to regulate decarbonized gas grids. In particular, there is, to the best of the author's knowledge, a lack of literature on gas grid costs and end customers tariff schemes. The need for more research on the regulation of gas grids in the future is however mentioned in several studies already. Khatiwada et al. \cite{khatiwada2022decarbonization} emphasize that the energy system decarbonization requires new rules and regulation of gas grids as well as restructuring of gas markets. Erdener \cite{erdener2023review} reviews literature on the regulation of gas grids with focus on the blending of hydrogen. Where overall, there is a growing trend for gas grid operators and regulators to look beyond short-term forecasts of gas grid tariffs to long-term forecasts (e.g., up to 2050), recently, the European Commission published a proposal on markets for renewable and natural gases and for hydrogen \cite{propsal_gas_market}. In this context, the report of the French Energy Regulatory \cite{french} deals with the French gas grid in the context of decarbonized energy systems in the years 2030 and 2050. Bouacida et al. \cite{bouacida2022impacts} study the impact of the decarbonization on the gas grid costs in France and Germany. Zwickl-Bernhard et al. \cite{zwickl2023design} show the need for socialization of increasing gas grid costs among remaining end customers.\vspace{0.3cm}

In addition, the literature on the design of grid tariffs in decarbonized electricity grids, for example, can provide useful information, although of course they face a fundamentally different situation with a significant increase in demand and associated end customer numbers expected. Peterson and Ros \cite{peterson2018future} provide a broad discussion on the regulation of electricity grids in the future. Fulli et al. \cite{fulli2019change} elaborate on the impact of electricity grid regulatory on electricity markets. Morell Dameto et al. \cite{morell2020revisiting} study electricity grid tariffs in the context of the energy system decarbonization.

\subsection{Novelties \added{and progress beyond state of the art}}\label{novelties}
\replaced{Against the background of the literature discussed in the previous sections, the contribution of this paper can be summarized based on four distinctive points. To our knowledge, these have not been addressed in such detail in the existing literature.}{The novelties of the present paper with respect to the existing literature described above can be summarized as follows}:

\begin{itemize}
	\item With the possible development of gas pipeline lengths, transport volumes, and refurbishment investments as shown by examining four decarbonization scenarios ranging from massive electrification to continued strong use of natural gas based on renewable energy, a detailed techno-economic analysis of the Austrian natural gas grid up to 2040 is carried out under the assumption of decarbonizing the entire energy system.
	\item The proposed analysis emphasizes the spatial granularity in modeling the natural gas grid. More precisely, the Austrian gas grid is represented by 657 generation and demand nodes and 738 gas pipeline sections. In doing so, the analysis provides relevant insights for transmission pipelines (as most of the analyses of scientific researchers and other third parties do) and distribution pipelines at the high- and mid-pressure grid levels.
	\item The cost of using the decarbonized Austrian gas grid in 2040 for the end customer is given based on the average grid costs, taking into account the aging of the existing gas grid and the resulting need for replacement investments in pipelines, as well as the possibility of decommissioning parts of the gas grid that are no longer used.
	\item The methodological extension of an existing gas grid model by an alternative supply option (e.g., trucking and on-site gas storage), an aspect which will contribute to the expected discussion on the economic efficiency of existing natural gas grids as energy systems are decarbonized and demand for natural gas declines, allows investigating the techno-economic trade-off between the expected oversized and thus underutilized or even replaced gas pipelines of decarbonized gas grids and off-grid solutions. 
\end{itemize}