\section{State-of-the-art and progress beyond}\label{stateoftheart}
This section discusses relevant scientific literature in the field of this work. It is divided into three parts. First, Section \ref{import} deals with the global and cross-country dimension of natural and renewable gas trade. It focuses on the impact of the decarbonization on gas markets and discusses also intra-country gas supply with a high spatial granularity of a grid representation. Then, Section \ref{approaches} examines different approaches of modeling gas grids. Section \ref{tariffs} elaborates on the regulation of gas grids and especially on gas grid charges. Finally, Section \ref{novelties} highlights the novelties of this work. Due to the complexity of the topic and the associated magnitude of possible relevant literature, what is not part of the literature review is briefly discussed. \textcolor{red}{Hier beschreiben was nicht part ist.} % Literature that examines the entire energy system across sectors is not considered. Instead, those studies are considered that focus primarily on the gas sector, as is the case in the present work. Literature that deals with the gas sector but does not consider the network is also excluded.  

\subsection{Decarbonized gas markets and cross-country trade}\label{import}

In 2021, the European Commission has published a proposal for a framework of renewable and natural gases and for hydrogen \cite{regulation_renewable_gases}. The aim is to support renewable and low carbon gases (i.e., biogas, biomethane, renewable and low carbon hydrogen as well as synthetic methane) in Europe and to reach a share of two-third of gaseous fuels in 2050 energy mix. Further details on the definition of renewable and low carbon gases can be found in \cite{briefing_renewable_gases}. The remaining one-third of gaseous fuels in 2050 is expected to be still fossil natural gas, but in combination with carbon capture, storage and utilization. Today, renewable and low carbon gases have only a minor contribution to Europe's energy mix. Bertasini et al. \cite{bertasini2023decarbonization} give a critical overview of the contribution of renewable gases to the decarbonization of the European energy system and grids. Kolb et al. \cite{kolb2021scenarios} focus in their work on the integration of renewable gases into gas markets. In addition, the latter study provides also a comprehensive literature review on the topic of renewable gases. Lochner \cite{lochner2011identification} elaborates on the European gas market and the identification of congestions in the gas transmission grid. Gorre et al. \cite{gorre2019production} deal exhaustively with future renewable gas generation costs.\vspace{0.3cm}

A key role in the transition to renewable and low carbon gas markets has the existing gas infrastructure. On the hand, the repurposing of existing pipelines especially at the transmission grid level allow to build up a hydrogen grid, as proposed in the so-called "Hydrogen Backbone" \cite{hydrogen_backbone}. In this context, also the recently extended terminal capacities for liquified natural gas (LNG) are worth to be mentioned. In the short-term, LNG terminals are used to support Russian natural gas import substitution by fossil LNG imports from exporter countries, such as the United States and Quatar \cite{brauers2021liquefied}. But in the mid-term, these terminals can be used to import renewable and low carbon gases, supporting the European gas market \cite{al2022emerging}. On the other hand, the area-wide existing pipelines of the distribution grid levels (high-, mid-, and low-pressure pipelines) allow the injection of distributed renewable and low carbon gas generation \cite{cucchiella2018profitability}. Sulewski \cite{sulewski2023development} explore the biomethane market in Europe. Schlund and Schönfisch \cite{schlund2021analysing} analyze the impact of renewable quota on the European natural gas markets. Paturska et al. \cite{paturska2015economic} provide an economic assessment of biomethane supply system based on the natural gas grid. Khatiwada \cite{khatiwada2022decarbonization} elaborate on barriers of the decarbonization of natural gas systems. Stürmer \cite{sturmer2020greening} examines in detail on the potentials of renewable gas injection into existing gas grids. 

\subsection{Gas grid modeling approach (top-down and bottom-up)}\label{approaches}
The following literature review focuses on the modeling of natural gas transport by grids and pipelines. There are other ways of transporting natural gas. The interested reader is referred to Thomas and Dawe \cite{thomas2003review} for a comprehensive review of the options for transporting natural gas. In general, the literature on gas grid modeling approaches can be divided based on two key dimensions: (i) modeling perspective (e.g., techno-economic) and (ii) spatial scale. These dimensions, along with others such as the sectoral dimension (whether or not hydrogen is accounted for in detail), determine the level of consideration given to various factors such as flow conditions of natural gas, pressure levels and drops in transport pipelines, and the operational energy and costs associated with compressors.\vspace{0.3cm}

A review on optimization of natural gas transportation systems is given by R{\'\i}os-Mercado and Borraz-S{\'a}nchez \cite{rios2015optimization}. It encompasses both transmission and distribution grids. Pfetsch et al. \cite{pfetsch2015validation} elaborate in detail on the operation of gas transmission grids. Pambour et al. \cite{pambour2016integrated} propose an integrated transient model approach for simulating the operation of transmission grids. The transient process in transmission grids is further examined by Liu \cite{liu2011coordinated}. Riepin et al. \cite{riepin2022adaptive} develop in their study an adaptive robust optimzation model for transmission grid expansion planning. Chiang and Zavala \cite{chiang2016large} investigate the interconnection between gas and power transmission grids. O'Donoghue et al. \cite{o1997development} examine transmission pipelines' resistance to high-pressure levels. Liu et al. \cite{liu2009security} study aspects of supply security in detail.\vspace{0.3cm}

With regard to the distribution grid level, Herr{\'a}n-Gonz{\'a}lez et al. \cite{herran2009modeling} provide a comprehensive review on the modeling and simulation of gas grids. Barati et al. \cite{barati2014multi} propose an integrated framework for grid expansion planning.  Giehl et al. \cite{giehl2023assessment} examine the impact of the decarbonization on gas distribution grids. Zwickl-Bernhard and Auer \cite{zwickl2022demystifying} present alternative supply options to natural gas distribution grids. Keogh et al. \cite{keogh2022gas} review technical and modeling studies of renewable gas generation and injection into the distribution grid. The same authors present also a techno-economic case study for renewable gas injection into the distribution grid in \cite{keogh2022gas}. Abeysekera et al. \cite{abeysekera2016steady} analyze the injection of renewable gas in low-pressure gas grids from a technical perspective in detail. Mertins et al. \cite{mertins2023competition} examine the competition between renewable gas and hydrogen injection into distribution grids. Repurposing of natural gas pipelines for hydrogen transport is assessed by Cerniauskas et al. \cite{cerniauskas2020options}. An overview of the modeling of hydrogen grids is given by Reuß et al. \cite{reuss2019modeling}.\vspace{0.3cm}

Finally, the modeling contributions of the open-source community subject of gas grids are discussed. In principle, open-source approaches are becoming increasingly important in energy system analysis \cite{hulk2018transparency}. This trend is also continuing in the area of gas grids. For instance, Schmidt et al. \cite{schmidt2017gaslib} provide a set of publicly available gas grid instances that can be used by researchers in the field of gas transport. Pluta et al. \cite{pluta2022scigrid_gas} present an approach for developing an open-source model of the gas transport grid in Europe. Nevertheless, data on natural gas grids in particular are rarely made publicly available. There are isolated exceptions, e.g. for the transmission grid (see \cite{entsog} for open-source data on the European transmission gas grid) or for the Belgian gas grid in \cite{de2000gas}. However, there is often an information advantage for those who have this information (e.g., gas grid operators) to scientific researchers, particularly with analyses at the distribution grid level. 

\subsection{Decarbonized gas grid regulation}\label{tariffs}
% Sehr wenig Literatur dazu wie die Regulierung bzw. Endkundentarifgestaltung in dekarbonisierten Gasnetzen aussehen kann. 
% Deutlich mehr Literatur besteht im Zusammenhang mit Stromnetzen auf dem Weg Richtung decarbonized.
% Hier 2-3 Literaturstellen zu Power Grid Tariff Renewables erwähnen.
% sehr wichtig weil auch tariff regulation long-term estimates until 2050; renewable and low-carbon gases
% dann hier diesen französischen bericht zitieren.
% Bouacida et al. \cite{bouacida2022impacts}: Impacts of greenhouse gas neutrality strategies on gas infrastructure and costs: implications from case studies based on French and German GHG-neutral scenarios
% dann das eigene paper zitieren wo von sozialisierung der kosten gesprochen wird. 






\subsection{Novelties}\label{novelties}
