\section{Introduction}
For decades in Europe, the optimal method of distributing natural gas to end customers, regardless of their varying demand scales (ranging from industrial facilities to individual households), has been through gas pipelines and gas grids \cite{rajnauth2008gas}. The first of two reasons for this is that natural gas has been a cheap energy source due to its virtually unlimited availability in Europe through imports, mainly from neighboring regions \cite{bilgin2009geopolitics}, with the second reason allowing large quantities of natural gas to provide various energy services, being that transporting natural gas through pipelines has been technically efficient and economically cheap over short and long distances \cite{thomas2003review}. This paper aims, among other things, to analyze how these gas grid costs for end customers could develop during decarbonizing energy systems in the future. It is seen that both reasons mentioned were also responsible for the fact that gas customers were only charged low tariffs for using the gas grid (historically mainly for withdrawal of natural gas).\vspace{0.3cm}
 
Austria was one of the first Western European countries connected to natural gas pipelines, having a long tradition of piped gas supply. The "Trans Austria Gas Pipeline" (TAG) started operation in 1968 and connected Austria with Slovakia \cite{gas_connect_austria}. Russian gas was transported. Where natural gas grids face an uncertain future, as does the Austrian gas grid, the outcomes of the long tradition of natural gas in Austria are reflected on the one hand in a high dependence on natural gas for the provision of energy services \cite{eurostat_natural_gas} and on the other hand in a well-developed gas grid in the country \cite{econtrol_grid}. European and national decarbonization policies, in a massive reduction in demand for natural gas, expected for the future in Europe \cite{repowereu}, are pushing the use of natural gas toward renewable energy alternatives in all energy sectors and services, making it unclear as to what extent gas grids will still be needed and whether they can be operated economically.\vspace{0.3cm}

Both reasons for efficient future gas grid operation mentioned above are questioned when considering the decline in demand for natural gas, carbon pricing, and the general shift toward electrification of energy services. The main objective of this paper is to contribute to this discussion by quantifying the scope and size of the Austrian gas grid, laying in the geographical center of the European gas grid, until 2040 under different decarbonization scenarios. In particular, the goal is to answer the following three research questions:

\begin{itemize}
	\item How does Austria's gas grid develop by 2040 under different decarbonization scenarios of the Austrian and European energy systems, ranging from electrification of most energy services to importing large amounts of renewable methane?
	\item Given the aging nature of gas grids and pipelines, what is the need for replacement investment in the Austrian gas grid by 2040, especially given the expected increase in renewable gas generation (e.g., biomethane and synthetic gas) and its gas grid injection?
	\item How does Austria's gas grid change by 2040 regarding grid costs for the end customer compared to the status quo?
\end{itemize}

The proposed analysis of the Austrian gas grid, the relevance of which must also be seen from a European perspective, is not only a detailed regional case study but also provides relevant insights for other countries with the expectation of a high potential for domestic renewable gas generation in the future, such as Germany, Italy, and France (see in \cite{scarlat2018biogas}). While changes in the Austrian gas grid might also impact the gas grid of neighboring countries and vice versa, the Austrian gas grid has historically been an important hub for the transmission and distribution of imported natural gas through Europe and provides ample storage capacities (see in \cite{sesini2021strategic}).\vspace{0.3cm}

A mixed-integer linear optimization approach is proposed to answer the three research questions. The applied model considers the existing natural gas grid (transmission, high-pressure, and mid-pressure pipelines) as a starting point and decides whether or not the gas grid supplies the gas demand and collects renewable gas generation. Alternatively, unmet demand and non-injected generation are considered to be met by the alternative transport option of trucking. The model considers the existing pipelines' age and the necessary replacement investments if they reach their technical lifetime and the option of early decommissioning in case of no or insufficient use of pipelines to reduce grid operating costs. 

Based on scenarios developed for a decarbonized Austrian energy system 2040 by the \textit{Environment Agency Austria} \cite{umweltbundesamt} and \textit{Austrian Energy Agency} \cite{Energieagentur}, the four different scenarios studied ("Electrification", "Green Gases", "Decentralized Green Gases", and "Green Methane") ensure the robustness of the analysis while covering a wide range of possible future gas volume developments in demand, imports, exports, and generation of gas. Therefore, the scenarios and work must be understood from a "what-if" perspective. Based on the scenarios determining the shares of the Austrian energy system's renewable/natural gas, hydrogen, power, and other energy carriers, the need for pipelines to transport and balance gas demand and generation is analyzed. Although no blending is considered, which is in line with the decarbonization strategy of the Austrian government, the reasonableness of blending is discussed in future work. Explicitly, no integrated energy system modeling across energy sectors/carriers or analysis of how fossil fuel-based energy services are decarbonized in detail is conducted.\vspace{0.3cm}

Regarding the third research question, some clarifications should be made so the reader can understand the context of grid costs, pricing, and tariffs. Essentially, the third research question asks how the gas grid tariff for end customers will develop under the assumption of a decarbonized gas sector and gas network, as, generally, gas grids and their tariff design, which is set by the regulatory authority for a period and is a complex process in which many different influencing factors are considered (see for example \cite{chen2020dynamic} and \cite{klein1993comparison}), are regulated with corresponding regulation periods (in the range of 5 years). The running costs of the grid are, among others, a key influencing factor. While a detailed study of the costs, prices, and tariffs of gas grids, and in particular, how these are interrelated, is undoubtedly relevant, it is beyond the scope of this paper. Against this background and given the cost-minimizing model approach, this paper takes a simplified approach to determining tariffs and simplifies the assumption that the end customer tariff is based solely on the running (average) gas grid costs. Saying this, the admittedly simplistic approach of moving from average grid costs to end customer's tariffs prompts us to continue referring to average grid costs.\vspace{0.3cm}  

Finally, for the sake of clarity, the terminology used is for gases: natural gas, renewable gas, biomethane, synthetic gas, renewable methane, and hydrogen. The term natural gas is essentially used when demand is meant or no distinction is necessary with regard to the energy source used. The introduction and use of the other terms, especially biomethane and synthetic gas, are motivated by the fact that this analysis is based on national studies and scenarios, with these underlying studies and scenarios precisely using these terms to respect the different potentials for biomethane and synthetic gas. The sum of both is then named renewable gas here, with renewable methane used when natural gas based on renewable energy is imported from neighboring countries, and in a few places in the paper where it is appropriate to do so, there is explicit mention of fossil natural gas. For a detailed discussion of the topic regarding the terminology of renewable gases, the reader is referred to recent papers \cite{ridjan2016terminology} and \cite{legendre2023state} as examples.\vspace{0.3cm}

The paper is organized as follows. Section \ref{stateoftheart} provides relevant literature and background information on the topic as well as the novelties of this work. Section \ref{methodology} explains the applied method and the four scenarios in detail. Section \ref{results} presents the results of the work, while Section \ref{synthesis} provides a synthesis of key findings. Section \ref{conclusions} concludes and outlines future research.