\section{Introduction}

In Europe, the most efficient way of transporting natural gas has been piped and grid-related transport for decades. There are two main reasons for this. First, natural gas has been an extremely cheap energy source due to its unlimited availability through imports from bordering regions. Hence, large quantities of natural gas have been demanded to cover various energy services. Second, the transport of natural gas through pipelines over short and long distances was and is technically efficient and economically cheap because of good flow conditions regarding pressure levels and, thus, transport capacities. In the context of piped natural gas supply, Austria has a long tradition. Austria was one of the first Western European countries connected to natural gas pipelines. The "Trans Austria Gas Pipeline" (TAG) started operation in 1968 and connected Austria with Slovakia. The effects or consequences of this long history of natural gas in Austria are reflected in a high dependence on natural gas in providing energy services and a well-developed natural gas grid throughout the country.\vspace{0.3cm}

However, natural gas grids face an uncertain future, as does the Austrian gas grid. European and national decarbonisation policies are pushing the use of natural gas towards renewable energy alternatives in all energy services. The consequence is a massive reduction in demand for natural gas. It is therefore unclear to what extent gas grids will still be needed. The main objective is to contribute to this discussion by quantifying the scope and size of the Austrian gas grid by 2040 under different decarbonization scenarios. The goal it to answer the following two research questions:

\begin{itemize}
	\item How does Austria's natural gas grid today develop from today to 2040 under different decarbonization scenarios, ranging from electrification of most of energy services to importing renewable methane?
	\item How do customer grid charges change by 2040 in a gas grid with a dominant supply of domestic renewable gas generation, such as the Austrian grid, while natural gas demand is declining?
\end{itemize}

\textcolor{red}{1) Wie wird diese Frage genau beantwortet; 2) Für wen ist diese Arbeit hilfreich. 3) Was wird nicht gemacht. 4) Organizer of the paper.}










das ist sicher ein genereller trend und trasferabel do all the european countries
in einigen ländern, wie deutschland, italien und auch österreich, wird die renewable gas production eine rolle spielen.
diese ist inbesondere in der fläche zu erwarten, könnte zu einem renewable betrieb von existierenden gasmetzen führen.
gleichzeitig würde das zu einem netz in der fläche führen 
außerdem spielt natürlich das thema hydrogen noch eine wichtige rolle. dessen rolle auf die gasnetze muss vermutlich unterschiedlich auf den netzebene gesehenm werden.
auf der fernleitungsebene zeichnet sich ein sehr klares bild ab
umwidmung da man stränge hat 
hydrogen backbone etc.
auf den untergelagerten netzebene (high. mid, und low pressure) ist die entwicklung unklar 
nicht nur wo die verbräuche sitzen weiterbetrieb von leitungen, sondern auch durch die einspeisung



% 208 Anlagen Biomethan, mit einer eingespeisten Energiemenge von 9.591 GWh/a [7]. aus dem IEWT Paper kopiert

% Die Einspeisung von Biomethan in das Gasnetz ist in Ländern wie Deutschland, Schweden und Österreich bereits gängige Praxis [17].

% Die Mindestgröße von Biogasaufbereitungsanlagen (250 Nm 3 /h) für eine wirtschaftliche Aufbereitung von Biogas zu Biomethan [21] übersteigt in der Regel die Größe der Biogasproduktionskapazität einzelner Biogasfermenter, die in Deutschland durchschnittlich 180 Nm 3 /h beträgt [22].

% Different components of a gas system, such as underground and over ground storage, gas turbines and engines, domestic and indus trialappliances, compressors, and valves, are usually designed to transport and operate on natural gas with a consistent quality. Therefore, at present it is not possible to specify limiting values for alternative gas injections which would be valid for all parts of the gas infrastructure.








%The paper is organized as follows. Section \ref{stateoftheart} provides an overview of the current state-of-the-art in scientific literature and outlines the novelties of this work beyond existing research. Section \ref{methodology} presents the materials and methods developed in this work, including, the model's mathematical formulation and description of different model runs. Section \ref{results} presents the results of this work encompassing different handlings of gas demands within the network. Section \ref{conclusions} synthesizes and discusses the results, concludes the work, and gives an outlook for future research.  