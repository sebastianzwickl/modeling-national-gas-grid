\section{Introduction}
For decades in Europe, the optimal method of distributing natural gas to end customers, regardless of their varying demand scales (ranging from large industrial facilities to individual households), has been consistently been through the utilization of pipelines and comprehensive gas grids \cite{rajnauth2008gas}. There are two main reasons for this. Firstly, natural gas has been a cheap energy source due to its unlimited availability in Europe through imports, mainly from neighbouring regions \cite{bilgin2009geopolitics}. And secondly, the transport of natural gas through pipelines has been technically efficient and economically cheap over both short and long distances \cite{thomas2003review}. Particularly the latter reason allowed for large quantities of natural gas used to provide various energy services throughout the territory. Both reasons mentioned were mainly responsible also for the fact that gas customers were only charged low costs for using the gas grid (historically mainly for withdrawals of natural gas, not or less for injections). This paper aims, among other things, to analyze how these gas grid costs for end customers could develop in the course of decarbonizing energy systems.\vspace{0.3cm}
 
In the context of piped natural gas supply, Austria has a long tradition. In fact, Austria was one of the first Western European countries connected to natural gas pipelines. The "Trans Austria Gas Pipeline" (TAG) started operation in 1968 and connected Austria with Slovakia \cite{gas_connect_austria}. The gas came from Russia. The consequences of this long history of natural gas in Austria are reflected on the one hand in a high dependence on natural gas for the provision of energy services \cite{eurostat_natural_gas} and on the other hand in a well-developed gas grid in the country \cite{econtrol_grid}. However, natural gas grids face an uncertain future, as does the Austrian gas grid. European and national decarbonization policies are pushing the use of natural gas towards renewable energy alternatives in all energy sectors and services. The consequence is a massive reduction in demand for natural gas expected for the future in Europe \cite{repowereu}. It is therefore unclear to what extent gas grids will still be needed and whether they can be operated economically. With reference to the first paragraph, both reasons for efficient gas grids are called into question when considering the decline in demand for natural gas, carbon pricing and the general shift towards electrification of energy services. The main objective of this paper is to contribute to this discussion by quantifying the scope and size of the Austrian gas grid, laying in the geographical center of the European gas grid, until 2040 under different decarbonization scenarios. In particular, the goal it to answer the following three research questions:

\begin{itemize}
	\item How does Austria's gas grid develop by 2040 under different decarbonization scenarios of the Austrian and European energy system, ranging from electrification of most of energy services to importing large amounts of renewable methane?
	\item Given the ageing nature of gas grids and pipelines, what is the need for replacement investment in the Austrian gas grid by 2040, especially in view of the expected increase in renewable gas generation (e.g., biomethane and synthetic gas) and its gas grid injection?
	\item How does Austria's gas grid change by 2040 in terms of grid costs for the end customer in comparison to the status quo?
\end{itemize}

The proposed analysis of the Austrian gas grid is not only a detailed regional case study, but also provides relevant insights for other countries with the expectation of a high potential for domestic renewable gas generation in the future, such as Germany, Italy, and France (see in \cite{scarlat2018biogas}). The relevance of this case study must also be considered from a European perspective. The Austrian gas grid has historically been an important hub for the transmission and distribution of imported natural gas through Europe and provides ample storage capacities (see in \cite{sesini2021strategic}). Therefore, changes in the Austrian gas grid might also impact the gas grid of neighboring countries and vice versa.\vspace{0.3cm}

A mixed-integer linear optimization approach is proposed to answer the three research questions. The applied model takes into account the existing natural gas grid (transmission, high-pressure and mid-pressure pipelines) as a starting point and decides whether or not the gas grid supplies the gas demand and collects renewable gas generation. Alternatively, unmet demand and uninjected generation are considered to be met by the alternative transport option of trucking. The model considers the existing pipelines' age and the necessary replacement investments if they reach their technical lifetime and the option of early decommissioning in case of no or insufficient use of pipelines to reduce grid operating costs. The four different scenarios studied ("Electrification", "Green Gases", "Decentralized Green Gases", and "Green Methane") ensure robustness of the analysis while covering a wide range of possible future gas volume developments in demand, imports, exports, and generation of gas. They base on scenarios developed for a decarbonized Austrian energy system 2040 by the \textit{Environment Agency Austria} \cite{umweltbundesamt} and \textit{Austrian Energy Agency} \cite{Energieagentur}. Therefore, the scenarios and work must be understood from a "what-if" perspective. The scenarios determine the shares of renewable/natural gas, hydrogen, power, and other energy carriers in the Austrian energy system. Based on that, the need for pipelines to transport and balance gas demand and generation is analyzed. No blending is considered. Explicitly, no integrated energy system modeling across energy sectors/carriers or analysis of how fossil fuel-based energy services are decarbonized in detail is conducted.\vspace{0.3cm}

In addition, for the sake of clarity, the terminology used in this paper should be briefly explained here. In general, the following terms are used for gases: natural gas, renewable gas, biomethane, synthetic gas, and hydrogen. In this work, natural gas is a fossil fuel, while all the others are renewable-based. The introduction and use of the other terms, especially biomethane and synthetic gas, are motivated by the fact that this analysis here is based on national studies and scenarios. These underlying studies and scenarios use exactly these terms in order to precisely respect the different potentials for biomethane and syntethic gas. The sum of both is then named as renewable gas here. In a few places in the paper where it is appropriate to do so, there is explicit mention of fossil natural gas. For a detailed dicussion of the topic regarding the terminology of renewable gases, the reader is referred to recent papers \cite{ridjan2016terminology} and \cite{legendre2023state} as examples.\vspace{0.3cm}

The paper is organized as follows. Section \ref{stateoftheart} provides relevant literature and background information on the topic as well as the novelties of this work. Section \ref{methodology} explains the applied method and the four scenarios in detail. Section \ref{results} present the results of the work, while Section \ref{synthesis} provides a synthesis of key findings. Section \ref{conclusions} concludes and outlines future research.