\section{Synthesis}\label{synthesis}
To the three research questions posed in this paper, the generated results show some expected and some unexpected effects. As expected, although the shrinking extent varies between the decarbonization scenarios, but generally is significantly lower than expected when looking solely at the demand, by looking at the assumed future volumes for the natural gas demand and renewable gas generation, the Austrian gas grid will shrink in a decarbonized energy system. The main driver for this probably unexpected result is the integration of decentralized renewable gas generation (biomethane and synthetic gas) and that stand-alone supply options (trucking and on-site gas storage) are not competitive with piped supply. In terms of grid costs, it is primarily the fixed costs of the existing gas grid (rather than the capital costs of the refurbished gas pipelines) that lead to, in some scenarios, a significant increase in average grid costs compared to the status quo (e.g., a fivefold increase in the scenario with high electrification of the energy system). Only in the scenario with continued high use of natural gas (through imports of renewable methane) do average gas grid costs remain similar to today's gas grids. An increase in end customers grid tariffs in line with grid costs can then be expected as a further consequence.\vspace{0.3cm}

Considering the ambitious national climate targets, their applicability extends to countries with similarly high aspirations for renewable gas generation, applying to cases such as the decarbonization of the gas sector, as per the findings above, and the overall results, with those for countries such as Germany, Italy, and France maybe looking similar in Europe. With the specific geographical location of the renewable gas and demand in the analysis having been proven to be too determining and crucial, these generalizations are more to be understood as qualitative statements and would require detailed analyses in any case.\vspace{0.3cm}

Concerning the study's limitations, two aspects should be considered when interpreting the results. FTreating gas grid costs and end customer tariffs is relatively simplistic and could mislead the inattentive reader, where, firstly, the results are primarily scenario-driven. For example, natural gas demand and renewable gas generation are determined by the scenarios and then used exogenously in the gas grid modeling. The demand and generation volumes are inelastic to gas grid costs. Secondly, based on the gas grid costs, an indication of the end customer tariff is given. Again, the average grid costs are used to give a quantitative indication of how grid tariffs for end customers may develop in the future. As always with this type of analysis, the number of assumptions that have to be made due to lack of information by the researcher and third parties should be taken into account when interpreting the present results, especially when dealing with sensitive data of the existing energy system, such as gas grid information.  




























