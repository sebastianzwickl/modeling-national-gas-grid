\section{Synthesis}\label{synthesis}
With respect to the three research questions posed in this paper, the generated results show some expected and some unexpected results. As expected, by also looking at the future demand volumes of natural and renewable gas, the Austrian gas grid in a decarbonized energy system will shrink. However, the extent of shrinking, varies though between the decarbonization scenarios, but is generally significantly lower than expected when looking solely at the future demand volumes. Main driver is the integration of decentralized renewable gas generation (biomethane and synthetic gas) and the fact that stand-alone supply options (trucking and on-site gas storage) are not competitive with piped supply. Nevertheless, in terms of grid costs, it is primarily the fixed costs of the existing gas grid (rather than the capital costs of the refurbished gas pipelines) that lead to a, in some scenarios, significant increase in average grid costs compared to the status quo (e.g., fivefold increase in the scenario with a high electrification of the energy system). Only in the scenario with continued high use of natural gas (trough imports of decarbonized natural gas) do average gas grid costs remain similar to those of today's gas grids.\vspace{0.3cm}

Assuming ambitious national climate targets (e.g. decarbonization of the gas sector), the findings discussed above and the results obtained in general can be generalized in the sense that they are valid for those countries with a similarly high expectation for renewable gas generation. In Europe, for instance, it is likely that the results for countries such as Germany, Italy and France might look similar. These generalizations are, of course, more to be understood as qualitative statements and would require detailed analyses in any case. The specific geographical location of the renewable gas (and demand) in the analysis have proven to be too determining and crucial.\vspace{0.3cm}

With regard to the limitations of the study, two aspects should be mentioned and taken into account when interpreting the results. First, the results are largely scenario driven. For example, natural gas demand and renewable gas generation are determined by the scenarios and then used exogenously in the gas network modeling. In essence, the demand and generation volumes are inelastic to gas network costs. Second, based on the gas network costs, an indication of the end customer costs is given. In this context, the treatment of (average) gas network and retail costs is relatively simplistic and could mislead the inattentive reader. Again, the average network costs are used to give a quantitative indication of how network costs for retail customers may develop in the future. As always with this type of analysis, especially when dealing with sensitive data of the existing energy system, such as gas network information, the number of assumptions that have to be made due to lack of information by the researcher and third parties should be taken into account when interpreting the present results. 
% Regarding the limitations of the study, among others, two aspects should be mentioned and taken into account when interpreting the results obtained. Firstly, the results are to a large extent scenario driven. For example, natural gas demand and renewable gas generation are determined by the scenarios and then used in the gas grid modeling exogenously. Essentially, the demand and generation volumes are inelastic to the gas grid costs. Secondly, based on the gas grid costs, an indication is given for end customers' costs. In this context, the way (average) gas grid and end customers' costs are treated is comparatively simplistic and could mislead the inattentive reader. Again, the average grid costs are used to give an quantitative indication of how the grid costs for end customers may develop in the future. As always in these kind of analyses, in particular when it comes to sensitive data of the existing energy system, such as gas grid information, the bunch of assumptions that have to be made because of a lack of information as a researcher and third-party should be taken into account when interpreting the present results. 





























