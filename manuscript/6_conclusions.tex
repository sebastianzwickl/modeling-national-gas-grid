\section{Conclusions}\label{conclusions}
The future of natural gas grids is one of the most pressing issues in realizing energy system decarbonization. This paper conducts a techno-economic analysis of the Austrian gas grid to 2040, a gas grid confronted with an expected significant decrease in natural gas demand coupled with a significant increase in decentralized renewable gas generation.  

% Unbestritten ist, dass ZUKÜNFTIGE ERDGASNETZE WERDEN KLEINER SEIN; WIE KLEIN HÄNGT DAVON AB WIE SEHR BIOMETHAN INTEGRIERT WIRD

% setzt man auf biomethane große netze weiter gebraucht.

% NETZTARIFE STEIGEN AN; WER BLEIBT DANN NOCH AM ERDGAS HÄNGEN
% dabei kommt es weniger auf die absoluten mengen an, sondern die verteilte einspeisung ist eher eine ja/nein entscheidung
% schaffen regional/lokal biomethan, genau abgestimmt wo weiterhin verbrauch bleibt

% zukünftige arbeiten, diese regionalen cluster zu identifizieren
% weitere technische details berücksichtigen, wie die druckentwicklung in schwächer ausgelasteten netzen, energie die gebraucht wird um druckhertzstellen, etc.





















