\section{Conclusions}\label{conclusions}
In many countries, the debate about using natural gas grids in sustainable energy systems has erupted, with the future of natural gas grids being one of the most pressing issues in realizing energy system decarbonization, at least in Europe. This paper contributes to the discussion by conducting a detailed national case study. While in particular, the case study is used to provide detailed insights into a well-developed gas grid with an expected significant decrease in natural gas demand and a significant increase in decentralized renewable gas generation, at the same time a techno-economic analysis of the Austrian gas grid to 2040 in four decarbonization scenarios is carried out.\vspace{0.3cm}

Austria's natural gas grids will shrink in the future; the natural gas demand will likely be spatially concentrated and restricted to large consumers, such as industrial facilities, and the level of shrinkage depends primarily on the level of integration of renewable gas and not on the level of demand for natural gas. The size of gas grids will be determined, on the one hand, by the quantities of domestic generation (and demand), on the other hand, by their spatial location. If an area-wide integration of domestic renewable gases into the gas grid happens, a significant increase in average grid costs and grid tariffs for the end customers must be expected. The aging of the existing gas grid and related refurbishment investments play a relatively minor role in the gas grid costs, as fixed costs mainly determine them. At the same time, off-grid solutions such as trucking and on-site storage are not competitive with the gas grid (even if the gas grid is very low utilized).\vspace{0.3cm}

The final finding on the increase in gas grid costs for large-scale renewable gas injection can be a starting point for further work. The questions that arise are not only who bears the high gas grid costs in such a case and what influence they have on the end customer's decision whether or not it is economical to stick with natural gas as an energy source, but also how synergies between renewable gas generators and natural gas demand can be exploited. The latter means exploring the spatial interplay of local generation and demand, for example, by forming regional renewable gas clusters. \added{One of the following questions is certainly how energy policy instruments can support these regional renewable gas clusters in an economically efficient way. At the same time, security of supply concerns need to be addressed in cases where these clusters are operated similarly to an islanded grid without significant supply redundancy (e.g., a regional renewable gas cluster in island mode with only one generation site could represent an extreme implementation).} Additionally, future research should examine the need for a dedicated hydrogen grid. That is a necessary complement to the present study, as hydrogen blending is not considered, and thus, hydrogen transport takes place in a separate grid if needed.\vspace{0.3cm}















